\documentclass{article}

\usepackage{amsmath}
\usepackage{circuitikz}

\title{Circuits}
\author{Tristan Slater}

\begin{document}
    \maketitle

    \tableofcontents

    \section{Components}

    \begin{table}[h]
        \begin{tabular}{llclc}
            \begin{circuitikz}[american]
                \draw (0, 0) to[R] (2, 0);
            \end{circuitikz} & Resistor & $R$ & Ohm & $\Omega$ \\
            \begin{circuitikz}[american]
                \draw (0, 0) to[C] (2, 0);
            \end{circuitikz} & Capacitor & $C$ & Farad & F \\
            \begin{circuitikz}[american]
                \draw (0, 0) to[L] (2, 0);
            \end{circuitikz} & Inductor & $L$ & Henry & H \\
            \begin{circuitikz}[american]
                \draw (0, 0) to[V] (2, 0);
            \end{circuitikz} & Voltage Source & $V$ & Henry & V \\
            \begin{circuitikz}[american]
                \draw (0, 0) to[I] (2, 0);
            \end{circuitikz} & Current Source & $I$ & Henry & A \\
        \end{tabular}
    \end{table}

    \section{Equivalent Resistance}

    \subsection{Series}

    \begin{equation}
        R_\text{eq} = \sum{R}
    \end{equation}

    \subsection{Parallel}

    \begin{equation}
        R_\text{eq} = \left[\sum{\frac{1}{R}}\right]^{-1}
    \end{equation}

    \section{Inductance}

    \begin{equation}
        v = L\frac{di}{dt}
    \end{equation}

    \section{Capacitance}

    \begin{equation}
        i = C\frac{dv}{dt}
    \end{equation}

    \section{Ohm's Law}

    \begin{equation}
        V = IR
    \end{equation}

    \section{Watt's Law}

    \begin{equation}
        P = IV
    \end{equation}

    \section{Kirchhoff's Laws}

    \subsection{Kirchhoff's Current Law (KCL)}

    For all lines in and out of a junction: \begin{equation}
        \sum{I} = 0
    \end{equation}

    \subsection{Kirchhoff's Voltage Law (KVL)}

    Around any closed loop circuit: \begin{equation}
        \sum{V} = 0
    \end{equation}

    \section{Nodal Analysis}

    \begin{itemize}
        \item Pick a reference node and treat as zero voltage (voltages are always relative)
        \item Use Kirchhoff's \textit{current} law (KCL) in terms of voltage
        \item Into node $\implies (+)$ 
        \item Out of node $\implies (-)$
    \end{itemize}

    \section{Loop Analysis}

    \begin{itemize}
        \item Use Kirchhoff's \textit{voltage} law (KVL) in terms of current
    \end{itemize}
\end{document}
